% Autogenerated translation of Getting-Started.md by Texpad
% To stop this file being overwritten during the typeset process, please move or remove this header

\documentclass[12pt]{book}
\usepackage{graphicx}
\usepackage[utf8]{inputenc}
\usepackage[a4paper,left=.5in,right=.5in,top=.3in,bottom=0.3in]{geometry}
\setlength\parindent{0pt}
\setlength{\parskip}{\baselineskip}
\renewcommand*\familydefault{\sfdefault}
\usepackage{hyperref}
\pagestyle{plain}
\begin{document}
\Large

\hrule
layout: global
title: 快速上手指南
group: User Guide

\section*{priority: 0}

\begin{itemize}
\item 内容列表
\{:toc\}
\end{itemize}

体验Alluxio最简单的方式是单机本地安装。在这个快速上手指南里,我们会引导你在本地机器上安装Alluxio,挂载样本数据,对Alluxio中的数据执行一些基本操作。具体来说,包括:

\begin{itemize}
\item 下载和配置Alluxio
\item 验证Alluxio运行环境
\item 本地启动Alluxio
\item 通过Alluxio Shell进行基本的文件操作
\item \textbf{[奖励]} 挂载一个公开的Amazon S3 bucket到Alluxio上
\item 关闭Alluxio
\end{itemize}

\textbf{[奖励]} 如果你有一个\href{http://docs.aws.amazon.com/AWSSimpleQueueService/latest/SQSGettingStartedGuide/AWSCredentials.html}{包含access key id和secret accsee key的AWS账户},你可以完成额外的任务。需要你AWS账户信息的章节都有\textbf{[奖励]}的标签。

\textbf{注意} 本指南旨在让你快速开始与Alluxio系统进行交互。Alluxio在大数据工作负载的分布式环境中表现最好。这些特性都难以适用于本地环境。如果你有兴趣运行一个更大规模的、能够突出Alluxio性能优势的例子,可以选择这两个白皮书中的任一个,尝试其中的指南:\href{https://alluxio.com/resources/accelerating-on-demand-data-analytics-with-alluxio}{Accelerating
on-demand data analytics with Alluxio}、\href{https://www.alluxio.com/blog/accelerating-data-analytics-on-ceph-object-storage-with-alluxio}{Accelerating data analytics on ceph object storage with Alluxio}。

\section*{前期准备}

为了接下来的快速上手指南,你需要:

\begin{itemize}
\item Mac OS X或Linux
\item \href{Java-Setup.html}{Java 8或更新版本}
\item \textbf{[奖励]} AWS账户和秘钥
\end{itemize}

\subsection*{安装SSH(Mac OS X)}

如果你使用Mac OS X,你必须能够ssh到localhost。远程登录开启方法:打开\textbf{系统偏好设置},然后打开\textbf{共享},确保\textbf{远程登录}已开启。

\section*{下载Alluxio}

首先,\href{http://www.alluxio.org/download}{下载Alluxio发布版本}。你可以从\href{http://www.alluxio.org/download}{Alluxio下载页面}下载最新的兼容不同Hadoop版本的\{\{site.ALLUXIO\emph{RELEASED}VERSION\}\}预编译版。

接着,你可以用如下命令解压下载包。取决于你下载的预编译二进制版本,你的文件名可能和下面有所不同。

\texttt{bash
\$ tar -xzf alluxio-\{\{site.ALLUXIO\_RELEASED\_VERSION\}\}-bin.tar.gz
\$ cd alluxio-\{\{site.ALLUXIO\_RELEASED\_VERSION\}\}
}

这会创建一个包含所有的Alluxio源文件和Java二进制文件的文件夹\texttt{alluxio-\{\{site.ALLUXIO\_RELEASED\_VERSION\}\}}。通过这个教程,这个文件夹的路径将被引用为\texttt{\$\{ALLUXIO\_HOME\}}。

\section*{配置Alluxio}

在开始使用Alluxio之前,我们需要配置它。大部分使用默认设置即可。

在\texttt{\$\{ALLUXIO\_HOME\}/conf}目录下,根据模板文件创建\texttt{conf/alluxio-site.properties}配置文件。

\texttt{bash
\$ cp conf/alluxio-site.properties.template conf/alluxio-site.properties
}

在\texttt{conf/alluxio-site.properties}文件中将 \texttt{alluxio.master.hostname}更新为你打算运行Alluxio Master的机器主机名。

\texttt{bash
\$ echo "alluxio.master.hostname=localhost" $>$$>$ conf/alluxio-site.properties
}

\subsection*{[奖励] AWS相关配置}

如果你有一个包含access key id和secret accsee key的AWS账户,你可以添加你的Alluxio配置以备接下来与Amazon S3的交互。如下命令可以添加你的AWS访问信息到\texttt{conf/alluxio-site.properties}文件。

\texttt{bash
\$ echo "aws.accessKeyId=$<$AWS\_ACCESS\_KEY\_ID$>$" $>$$>$ conf/alluxio-site.properties
\$ echo "aws.secretKey=$<$AWS\_SECRET\_ACCESS\_KEY$>$" $>$$>$ conf/alluxio-site.properties
}

你必须将\textbf{\texttt{$<$AWS\_ACCESS\_KEY\_ID$>$}}替换成你的AWS access key id,将\textbf{\texttt{$<$AWS\_SECRET\_ACCESS\_KEY$>$}}替换成你的AWS secret access key。现在,Alluxio完全配置好了。

\section*{验证Alluxio运行环境}

在启动Alluxio前,我们要保证当前系统环境下Alluxio可以正常运行。我们可以通过运行如下命令来验证Alluxio的本地运行环境:

\texttt{bash
\$ ./bin/alluxio validateEnv local
}

该命令将汇报在本地环境运行Alluxio可能出现的问题。如果你配置Alluxio运行在集群中,并且你想要验证所有节点的运行环境,你可以运行如下命令:

\texttt{bash
\$ ./bin/alluxio validateEnv all
}

我们也可以使用该命令运行某些特定验证项目。例如,

\texttt{bash
\$ ./bin/alluxio validateEnv local ulimit
}

将只运行验证本地系统资源限制方面的项目。

你可以在\href{Developer-Tips.html}{这里}查看更多关于本命令的信息。

\section*{启动Alluxio}

接下来,我们格式化Alluxio为启动Alluxio做准备。如下命令会格式化Alluxio的日志和worker存储目录,以便接下来启动master和worker。

\texttt{bash
\$ ./bin/alluxio format
}

现在,我们启动Alluxio!Alluxio默认配置成在localhost启动master和worker。我们可以用如下命令在localhost启动Alluxio:

\texttt{bash
\$ ./bin/alluxio-start.sh local SudoMount
}

恭喜!Alluxio已经启动并运行了!你可以访问\href{http://localhost:19999}{http://localhost:19999}查看Alluxio master的运行状态,访问\href{http://localhost:30000}{http://localhost:30000}查看Alluxio worker的运行状态。

\section*{使用Alluxio Shell}

既然Alluxio在运行,我们可以通过\href{Command-Line-Interface.html}{Alluxio shell}检查Alluxio文件系统。Alluxio shell包含多种与Alluxio交互的命令行操作。你可以通过如下命令调用Alluxio shell:

\texttt{bash
\$ ./bin/alluxio fs
}

该命令将打印可用的Alluxio命令行操作。

你可以通过\texttt{ls}命令列出Alluxio里的文件。比如列出根目录下所有文件:

\texttt{bash
\$ ./bin/alluxio fs ls /
}

不过现在Alluxio里没有文件。我们可以拷贝文件到Alluxio。\texttt{copyFromLocal}命令可以拷贝本地文件到Alluxio中。

\texttt{bash
\$ ./bin/alluxio fs copyFromLocal LICENSE /LICENSE
Copied LICENSE to /LICENSE
}

拷贝\texttt{LICENSE}文件之后,我们可以在Alluxio中看到它。列出Alluxio里的文件:

\texttt{bash
\$ ./bin/alluxio fs ls /
-rw-r--r-- staff  staff     26847 NOT\_PERSISTED 01-09-2018 15:24:37:088 100\% /LICENSE
}
输出显示\texttt{LICENSE}文件在Alluxio中,也包含一些其他的有用信息,比如文件的大小,创建的日期,文件的所有者和组,以及Alluxio中这个文件的占比。

你也可以通过Alluxio shell来查看文件的内容。\texttt{cat}命令可以打印文件的内容。

```bash
\$ ./bin/alluxio fs cat /LICENSE
                                 Apache License
                           Version 2.0, January 2004
                        http://www.apache.org/licenses/

   TERMS AND CONDITIONS FOR USE, REPRODUCTION, AND DISTRIBUTION
...
```

默认设置中,Alluxio使用本地文件系统作为底层文件系统(UFS)。默认的UFS路径是\texttt{./underFSStorage}。我们可以查看UFS中的内容:

\texttt{bash
\$ ls ./underFSStorage/
}

然而,这个目录不存在!这是由于Alluxio默认只写入数据到Alluxio存储空间,而不会写入UFS。

但是,我们可以告诉Alluxio将文件从Alluxio空间持久化到UFS。shell命令\texttt{persist}可以做到。

\texttt{bash
\$ ./bin/alluxio fs persist /LICENSE
persisted file /LICENSE with size 26847
}

如果我们现在再次检查UFS,文件就会出现。

\texttt{bash
\$ ls ./underFSStorage
LICENSE
}

如果我们在\href{http://localhost:19999/browse}{master webUI}中浏览Alluxio文件系统,我们可以看见\texttt{LICENSE}文件以及其它有用的信息。这里,\textbf{Persistence State}栏显示文件为\textbf{PERSISTED}。

\section*{[奖励]Alluxio中的挂载}

Alluxio通过统一命名空间的特性统一了对底层存储的访问。你可以阅读\href{http://www.alluxio.com/2016/04/unified-namespace-allowing-applications-to-access-data-anywhere/}{统一命名空间的博客}和\href{Unified-and-Transparent-Namespace.html}{统一命名空间的文档}获取更详细的解释。

这个特性允许用户挂载不同的存储系统到Alluxio命名空间中并且通过Alluxio命名空间无缝地跨存储系统访问文件。

首先,我们在Alluxio中创建一个目录作为挂载点。

\texttt{bash
\$ ./bin/alluxio fs mkdir /mnt
Successfully created directory /mnt
}

接着,我们挂载一个已有的S3 bucket样本到Alluxio。你可以使用我们提供的S3 bucket样本。

\texttt{bash
\$ ./bin/alluxio fs mount -readonly alluxio://localhost:19998/mnt/s3 s3a://alluxio-quick-start/data
Mounted s3a://alluxio-quick-start/data at alluxio://localhost:19998/mnt/s3
}

现在,S3 bucket已经挂载到Alluxio命名空间中了。

我们可以通过Alluxio命名空间列出S3中的文件。使用熟悉的\texttt{ls}shell命令列出S3挂载目录下的文件。

\texttt{bash
\$ ./bin/alluxio fs ls /mnt/s3
-r-x------ staff  staff    955610 PERSISTED 01-09-2018 16:35:00:882   0\% /mnt/s3/sample\_tweets\_1m.csv
-r-x------ staff  staff  10077271 PERSISTED 01-09-2018 16:35:00:910   0\% /mnt/s3/sample\_tweets\_10m.csv
-r-x------ staff  staff     89964 PERSISTED 01-09-2018 16:35:00:972   0\% /mnt/s3/sample\_tweets\_100k.csv
-r-x------ staff  staff 157046046 PERSISTED 01-09-2018 16:35:01:002   0\% /mnt/s3/sample\_tweets\_150m.csv
}

我们也可以\href{http://localhost:19999/browse?path=%2Fmnt%2Fs3}{在web UI上看见新挂载的文件和目录}。

通过Alluxio统一命名空间,你可以无缝地从不同存储系统中交互数据。举个例子,使用\texttt{ls}shell命令,你可以递归地列举出一个目录下的所有文件。

\texttt{bash
\$ ./bin/alluxio fs ls -R /
-rw-r--r-- staff  staff     26847 PERSISTED 01-09-2018 15:24:37:088 100\% /LICENSE
drwxr-xr-x staff  staff         1 PERSISTED 01-09-2018 16:05:59:547  DIR /mnt
dr-x------ staff  staff         4 PERSISTED 01-09-2018 16:34:55:362  DIR /mnt/s3
-r-x------ staff  staff    955610 PERSISTED 01-09-2018 16:35:00:882   0\% /mnt/s3/sample\_tweets\_1m.csv
-r-x------ staff  staff  10077271 PERSISTED 01-09-2018 16:35:00:910   0\% /mnt/s3/sample\_tweets\_10m.csv
-r-x------ staff  staff     89964 PERSISTED 01-09-2018 16:35:00:972   0\% /mnt/s3/sample\_tweets\_100k.csv
-r-x------ staff  staff 157046046 PERSISTED 01-09-2018 16:35:01:002   0\% /mnt/s3/sample\_tweets\_150m.csv
}

输出显示了Alluxio文件系统根目录下来源于挂载的不同文件系统的所有文件。\texttt{/LICENSE}文件在本地文件系统,\texttt{/mnt/s3/}目录下是S3的文件。

\section*{[奖励]用Alluxio加速数据访问}

由于Alluxio利用内存存储数据,它可以加速数据的访问。首先,我们看一看Alluxio中一个文件的状态(从S3中挂载)。

\texttt{bash
\$ ./bin/alluxio fs ls /mnt/s3/sample\_tweets\_150m.csv
-r-x------ staff  staff 157046046 PERSISTED 01-09-2018 16:35:01:002   0\% /mnt/s3/sample\_tweets\_150m.csv
}

输出显示了文件\textbf{Not In Memory}。该文件是微博的样本。我们看看有多少微博提到了单词“kitten”。使用如下的命令,我们可以统计含有“kitten”的tweet数量。

```bash
\$ time ./bin/alluxio fs cat /mnt/s3/sample\emph{tweets}150m.csv | grep -c kitten
889

real	0m22.857s
user	0m7.557s
sys	0m1.181s
```

取决于你的网络连接状况,该操作可能会超过20秒。如果读取文件时间过长,你可以选择一个小一点的数据集。该目录下的其他文件是该文件的更小子集。
如你所见,每个访问数据的命令都需要花费较长的时间。通过将数据放在内存中,Alluxio可以提高访问数据的速度。

在通过\texttt{cat}命令获取文件后,你可以用\texttt{ls}命令查看文件的状态:

\texttt{bash
\$ ./bin/alluxio fs ls /mnt/s3/sample\_tweets\_150m.csv
-r-x------ staff  staff 157046046 PERSISTED 01-09-2018 16:35:01:002 100\% /mnt/s3/sample\_tweets\_150m.csv
}

输出显示文件已经100\%被加载到Alluxio中,既然如此,那么再次访问该文件的速度应该会快很多。

让我们来统计一下拥有“puppy”这个单词的微博的数目。

```bash
\$ time ./bin/alluxio fs cat /mnt/s3/sample\emph{tweets}150m.csv | grep -c puppy
1553

real	0m1.917s
user	0m2.306s
sys	0m0.243s
```

如你所见,读文件的速度非常快,仅仅需要数秒钟!并且,因为数据已经存放到了Alluxio中了,你可以轻易的以较快的速度再次读该文件。
现在让我们来统计一下有多少微博包含“bunny”这个词。

```bash
\$ time ./bin/alluxio fs cat /mnt/s3/sample\emph{tweets}150m.csv | grep -c bunny
907

real	0m1.983s
user	0m2.362s
sys	0m0.240s
```

恭喜!你在本地安装了Alluxio并且通过Alluxio加速了数据访问!

\section*{中止Alluxio}

你完成了本地安装和使用Alluxio,你可以使用如下命令中止Alluxio:

\texttt{bash
\$ ./bin/alluxio-stop.sh local
}

\section*{总结}

恭喜你完成了Alluxio的快速上手指南!你成功地在本地电脑上下载和安装Alluxio,并且通过Alluxio shell进行了基本的交互。这是一个如何上手Alluxio的简单例子。

除此之外还有很多可以学习。你可以通过我们的文档学到Alluxio的各种特性。你可以在你的环境中安装Alluxio,挂载你已有的存储系统到Alluxio,配置你的应用和Alluxio一起工作。浏览下方以获得更多信息。

\subsection*{部署Alluxio}

Alluxio可以部署在很多不同的环境下。

\begin{itemize}
\item \href{Running-Alluxio-Locally.html}{本地运行Alluxio}
\item \href{Running-Alluxio-on-a-Cluster.html}{在集群上独立运行Alluxio}
\item \href{Running-Alluxio-on-Virtual-Box.html}{在Virtual Box中运行Alluxio}
\item \href{Running-Alluxio-On-Docker.html}{在Docker中运行Alluxio}
\item \href{Running-Alluxio-on-EC2.html}{在EC2上运行Alluxio}
\item \href{Running-Alluxio-on-GCE.html}{在GCE上运行Alluxio}
\item \href{Running-Alluxio-on-Mesos.html}{在EC2上使用Mesos运行Alluxio}
\item \href{Running-Alluxio-Fault-Tolerant-on-EC2.html}{在EC2上运行带容错机制的Alluxio}
\item \href{Running-Alluxio-Yarn-Integration.html}{Alluxio与YARN整合运行}
\item \href{Running-Alluxio-Yarn-Standalone.html}{在YARN Standalone模式上运行Alluxio}
\end{itemize}

\subsection*{底层存储系统}

有很多可以通过Alluxio访问的底层存储系统。

\begin{itemize}
\item \href{Configuring-Alluxio-with-Azure-Blob-Store.html}{Alluxio使用Azure Blob Store}
\item \href{Configuring-Alluxio-with-S3.html}{Alluxio使用S3}
\item \href{Configuring-Alluxio-with-GCS.html}{Alluxio使用GCS}
\item \href{Configuring-Alluxio-with-Minio.html}{Alluxio使用Minio}
\item \href{Configuring-Alluxio-with-Ceph.html}{Alluxio使用Ceph}
\item \href{Configuring-Alluxio-with-Swift.html}{Alluxio使用Swift}
\item \href{Configuring-Alluxio-with-GlusterFS.html}{Alluxio使用GlusterFS}
\item \href{Configuring-Alluxio-with-MapR-FS.html}{Alluxio使用MapR-FS}
\item \href{Configuring-Alluxio-with-HDFS.html}{Alluxio使用HDFS}
\item \href{Configuring-Alluxio-with-secure-HDFS.html}{Alluxio使用Secure HDFS}
\item \href{Configuring-Alluxio-with-OSS.html}{Alluxio使用OSS}
\item \href{Configuring-Alluxio-with-NFS.html}{Alluxio使用NFS}
\end{itemize}

\subsection*{框架和应用}

不同的框架和应用和Alluxio整合。

\begin{itemize}
\item \href{Running-Spark-on-Alluxio.html}{Apache Spark使用Alluxio}
\item \href{Running-Hadoop-MapReduce-on-Alluxio.html}{Apache Hadoop MapReduce使用Alluxio}
\item \href{Running-HBase-on-Alluxio.html}{Apache HBase使用Alluxio}
\item \href{Running-Flink-on-Alluxio.html}{Apache Flink使用Alluxio}
\item \href{Running-Presto-with-Alluxio.html}{Presto 使用 Alluxio}
\item \href{Running-Hive-with-Alluxio.html}{Apache Hive 使用 Alluxio}
\item \href{Accessing-Alluxio-from-Zeppelin.html}{Apache Zeppelin使用Alluxio}
\end{itemize}

\end{document}
